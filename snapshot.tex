\documentclass[10pt,a5paper,DIV=21]{scrartcl}

\AtBeginDocument{%
  \IfFileExists{iftex.sty}{%
    \RequirePackage{iftex}%
    \RequirePDFTeX%
  }{%
    \PackageWarning{snapshot}{%
      Please install the iftex package\MessageBreak
    }%
  }%
}

%\usepackage[top=12mm,bottom=26mm,outer=28mm,inner=14mm,foot=14mm]{geometry}
\usepackage[top=10mm,bottom=18mm,outer=12mm,inner=12mm,foot=9.5mm]{geometry}
\usepackage{calc}
\deffootnote[1.5em]{0em}{1em}{\thefootnotemark\quad}
\renewcommand{\footnoterule}{%
  \kern -2.4pt
  \hrule width \textwidth height 0.4pt
  \kern 2pt
}

\usepackage[T1]{fontenc}
\usepackage{textcomp}
\usepackage{helvet}

\usepackage{microtype,ellipsis}

\usepackage[ngerman]{babel}
%\usepackage{polyglossia}
%\setotherlanguage{russian} % the name of the original Russian version at the end of this book is written using Cyrillic letters


\usepackage{amsmath,amssymb,nicefrac,amscd}
\usepackage{graphicx,float}
\usepackage{pdfpages}

\usepackage{enumitem}
\setitemize[1]{noitemsep,nosep,leftmargin=0.99em,label={--}}

\usepackage{transparent}
\usepackage{csquotes}
\usepackage{siunitx}
\sisetup{per-mode=fraction,fraction-function=\nicefrac}
\DeclareSIUnit[number-unit-product=\,]\uhr{Uhr}
\DeclareSIUnit[number-unit-product=\,]\zoll{Zoll}

\usepackage{hyperref}

\usepackage{todonotes}

%\setdefaultlanguage{german}
\usepackage[math]{blindtext}
\usepackage{lipsum}
\usepackage{xcolor}
\usepackage{type1cm}
\usepackage{relsize}

\usepackage{scrpage2}
\pagestyle{scrheadings}
\clearscrheadfoot
\cfoot{\pagemark}
\newcommand{\arial}{\fontspec[Extension=.ttf, ItalicFont=Arial Italic]{Arial}}

\setkomafont{pagenumber}{\arial\relscale{0.9}}
\setkomafont{section}{\arial\relscale{1.1}\addfontfeature{LetterSpace=6.0}\underline }
\setkomafont{subsection}{\arial\relscale{0.9}\addfontfeature{LetterSpace=6.0}}
\setkomafont{subsubsection}{\arial\relscale{0.9}\addfontfeature{LetterSpace=6.0}}
\setkomafont{paragraph}{\arial\relscale{0.9}\addfontfeature{LetterSpace=6.0}}
\setkomafont{subparagraph}{\arial\relscale{0.9}\addfontfeature{LetterSpace=6.0}}
\setkomafont{captionlabel}{\arial\relscale{0.9}\addfontfeature{LetterSpace=6.0}}

\usepackage{tikz}
\usetikzlibrary{arrows}

\begin{document}
\noindent
\parbox[t]{0.5\linewidth}{\arial\relscale{0.75}\addfontfeature{LetterSpace=6.0}Schnappsch\"usse moderner Mathematik\\aus Oberwolfach}
\parbox[t]{0.5\linewidth}{\raggedleft\mbox{\textnumero \,1}}
\bigskip 
\bigskip 
\bigskip
\begin{center}

\textcolor{black!75!white}{\normalsize {\textscale{1.5}{{\arial\addfontfeature{LetterSpace=8.0}
	\parbox{0.75\linewidth}{%
		\centering
		 Ein toller SNAPSHOT mit einem ziemlich langen Titel der auch noch umbricht
		 \\
		\rule{2em}{0.075em}
		\\[\bigskipamount]
	}%
}}}}%

{\normalsize{\addfontfeature{LetterSpace=6.0}Torsten Voigt}}
\end{center}
\bigskip
\bigskip
\bigskip

\noindent\textscale{1.3}{\parbox{\linewidth}{\noindent Ein kurzer Text, der mal beschreibt, was so im nachfolgenden Text kommt. Gerne auch mal Abstract genannt. Bisschen l\"anger als zwei Zeilen sollte es aber schon sein. Vielleicht vier oder f\"unf. Eine Zeile geht noch, so dass wir hier wenigstens vier haben. Ein kurzer Text, der mal beschreibt, was so im nachfolgenden Text kommt.}}
\bigskip

\section{Eine \"Uberschrift}
\noindent\blindtext[1]\footnote{Eine Fu\ss note \"uber eine oder mehrere Zeilen, gesetzt nach Konrads W\"unschen.}
\begin{align}
	\rho \dot{\mathbf{v}}
	 &= \rho \left( \frac{\partial\mathbf{v}}{\partial t} + (\mathbf{v} \cdot \nabla) \mathbf{v} \right)\\
	 &=-\nabla p + \mu \Delta \mathbf{v} + (\lambda + \mu) \nabla (\nabla \cdot \mathbf{v})+\mathbf{f}.
\end{align}
Ein kurzer Text, der mal beschreibt, was so im nachfolgenden Text kommt. Gerne auch mal Abstract genannt. Bisschen l\"anger als zwei Zeilen sollte es aber schon sein. Vielleicht vier oder f\"unf. Eine Zeile geht noch, so dass wir hier wenigstens vier haben.

\section{Eine zweite \"Uberschrift g}
\subsection{Eine Unter\"uberschrift}
\blindtext

\blindtext

\subsection{Eine zweite Unter\"uberschrift}
\subsubsection{Eine Unterunter\"uberschrift}
\begin{figure}
\centering
\tikzstyle{int}=[draw, fill=black!20, minimum size=2em]
\tikzstyle{init} = [pin edge={to-,thin,black}]
\begin{tikzpicture}[node distance=2.5cm,auto,>=latex']
    \node [int, pin={[init]above:$v_0$}] (a) {$\frac{1}{s}$};
    \node (b) [left of=a,node distance=2cm, coordinate] {a};
    \node [int, pin={[init]above:$p_0$}] (c) [right of=a] {$\frac{1}{s}$};
    \node [coordinate] (end) [right of=c, node distance=2cm]{};
    \path[->] (b) edge node {$a$} (a);
    \path[->] (a) edge node {$v$} (c);
    \draw[->] (c) edge node {$p$} (end) ;
\end{tikzpicture}
\caption{Eine Bildunterschrift}
\end{figure}
\blindtext[2]
\paragraph{Absatztitel}\blindtext[1]
\subparagraph{Unterabsatztitel}\blindtext[1]
\begin{equation}
x^2+y^2+z^2-1+\mathbf{v}+\vec{v}=0
\end{equation}
\blindtext

\null\vfill
\noindent\begin{minipage}[b]{0.5\linewidth}
\raggedright
\arial
\addfontfeature{LetterSpace=6.0}
\footnotesize
Classification: Algebraic Geometry

Licence: by-nc-nd
\bigskip

Imaginary -- open mathematics\\
Mathematisches Forschungsinstitut\\
Oberwolfach gGmbH\\
Schwarzwaldstr. 9--11\\
D-77709 Oberwolfach-Walke, Germany
\end{minipage}%
\hspace{1ex}
\begin{minipage}[b]{0.45\linewidth}%
\raggedright
\arial
\addfontfeature{LetterSpace=6.0}
\footnotesize
MFO workshop title:\\
\textit{Classical and Quantum Mechanical Models of Many-Particle Systems}
\\[\medskipamount]

MFO Workshop Date:\\
1 Dec -- 7 Dec 2013
\\[\medskipamount]

Responsible Workshop Organizer:\\
Anton Arnold, Wien
Eric Carlen, Piscataway
Laurent Desvillettes, Cachan
\end{minipage}
\end{document}
%
% TODO:
%- sans font: arial
%- text letzte seite -> arial
%- seitenzahl -> nicht kursiv