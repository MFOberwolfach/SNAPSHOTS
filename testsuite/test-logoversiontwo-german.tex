\documentclass{snapshotmfo}

\categorizationmath{algebra and number theory,analysis,discrete mathematics and foundations,geometry and topology,numerics and scientific computing,probability theory and statistics} %at least one must be chosen. 
\categorizationconnect{chemistry and earth science,engineering and technology,finance,humanities and social sciences,life science,physics,reflections on mathematics} %can be void.
\license{CC-BY-SA-4.0} %recommended
\snapshotid{124}{1950}
\junioreditor[m]{Some One}{junior-editors@mfo.de}
\senioreditor[f]{Carla Cederbaum}{senior-editor@mfo.de}
\director[m]{Gerhard Huisken}
\usepackage[utf8]{inputenc}
\usepackage{amsmath,amssymb}

\usepackage[ngerman]{babel}
%\usepackage[USenglish]{babel}

\author{Test Autor}
\title{deutsches Logoband Version 2}
\begin{document}

\begin{abstract}[Sind auf der letzten Seite dieses Schnappschusses 2 bzw. 3 Logos zu sehen, u.a. die Worte \glqq Mitglied der Leibniz-Gemeinschaft\grqq ?]
Sind auf der letzten Seite dieses Schnappschusses 2 bzw.\ 3 Logos zu sehen, u.\,a. die Worte \glqq Mitglied der Leibniz-Gemeinschaft\grqq ?
\end{abstract}

\section{Kommentar zu diesem Komponententest}
Version 2 des Logobandes gegen Ende der letzten Seite zeigt folgende Logos:
\begin{enumerate}
  \item das MFO als Mitglied der Leibniz Gemeinschaft,
  \item Imaginary.
\end{enumerate}
Es ist standardmäßig aktiviert
und wird für Schnappschüsse verwendet, die ab dem Jahr 2017 eingereicht werden.

\end{document}
